\documentclass{article}
\usepackage[utf8]{inputenc}
\usepackage[english]{babel}
\usepackage[]{amsthm}
\usepackage[]{amsmath} 
\usepackage[]{amssymb} 
\usepackage{enumitem}
\usepackage{array}
\usepackage{mathtools}
\usepackage{gensymb}

\title{HW 4 - MATH411}
\author{Danesh Sivakumar}
\date{February 17, 2022}

\begin{document}
\maketitle 


\subsection*{Problem 1 (Exercise 1, Page 304)}

Determine which of the following subsets of $\mathbb{R}$ is sequentially compact. Justify your conclusions. 
\begin{enumerate}[label = \alph*)]
\item $\{x \text{ in } [0, 1] \mid x \text{ is rational} \}$
\item $\{x \text{ in } \mathbb{R} \mid x^2 > x \}$
\item $\{x \text{ in } \mathbb{R} \mid e^x-x^2 \leq 2 \}$
\end{enumerate}

\begin{proof} \quad \\
\begin{enumerate}[label = \alph*)]
\item This set is not sequentially compact; take $\{u_k\} = \frac{1}{3}(1+\frac{1}{k})^k$. Each $u_k$ is rational and in $[0, 1]$, but $\{u_k\} \to \frac{e}{3} \in [0, 1]$ , which is not rational, so it is not closed. Also, because $\mathbb{Q}$ is dense in $\mathbb{R}$, every point in $\mathbb{R}$ is a limit point of $\mathbb{Q}$, so that there exist sequences of rationals converging to an irrational; in particular, there exists sequences of rationals in $[0, 1]$ converging to an irrational in $[0, 1]$.
\item This set is not sequentially compact; take $\{u_k\} = 1+\frac{1}{k}$. Each $u_k$ is in the set, but $\{u_k\} \to 1$, which is not in the set, so it is not closed.
\item This set is not sequentially compact, because it is not bounded. To prove so, assume that it is bounded; that is, there exists $M > 0$ such that $|e^x - x^2| \leq M$. Take $x = -\sqrt{M}$; then it follows that $|e^{-\sqrt{M}} - ((-\sqrt{M})^2)| = |e^{-\sqrt{M}} - \sqrt{M}^2| > |0 - \sqrt{M}^2| = M$, a contradiction.     
\end{enumerate}
\end{proof}


\subsection*{Problem 2 (Exercise 6, Page 304)}

Let $A$ be a subset of $\mathbb{R}^n$ and let the function $f \colon A \to \mathbb{R}$ be continuous. 
\begin{enumerate}[label = \alph*)]
\item If $A$ is bounded, is $f(A)$ bounded?
\item If $A$ is closed, is $f(A)$ closed?
\end{enumerate}

\begin{proof} \quad \\
\begin{enumerate}[label = \alph*)]
\item No; consider $f \colon \mathbb{R} \to \mathbb{R}, f(x) = \frac{1}{x}$ on the set $A = (0, 1)$. Clearly, $A$ is bounded with $M = 1$ and $f$ is continuous by the continuity of quotients of continuous functions. However, $f(A)$ is not bounded; to prove this, suppose that there exists $M > 0$ such that $f(x) \leq M$ for all $x \in (0, 1)$. Consider $x = \frac{1}{M+1}$; then $f(x) = M+1 > M$, a contradiction.
\item No; consider $f \colon \mathbb{R} \to \mathbb{R}, f(x) = \frac{1}{1+x^2}$ on the set $A = [0, \infty)$. Note that $A$ is closed, but $f(A)$ is $(0, 1]$, which is not closed in $\mathbb{R}$ (consider the sequence $\{u_k\} = \frac{1}{k}$, which converges to $0 \notin A$.)
\end{enumerate}
\end{proof}

\subsection*{Problem 3 (Exercise 7, Page 304)}
Suppose that the function $f \colon \mathbb{R}^n \to \mathbb{R}$ is continuous and that $f(u) \geq ||u||$ for every point $u$ in $\mathbb{R}^n$. Prove that $f^{-1}([0, 1])$ is sequentially compact.

\begin{proof}
Note that because $[0, 1]$ is closed in $\mathbb{R}$ and $f \colon \mathbb{R}^n \to \mathbb{R}$ is continuous, it follows that $f^{-1}([0, 1])$ is closed in $\mathbb{R}^n$. Let $x \in [0, 1]$ be arbitrary. It follows that $0 \leq f(x) \leq 1$. Because $f(x) \geq ||x||$, we have that $||x|| \leq 1$, meaning that $f^{-1}([0, 1])$ is bounded in $\mathbb{R}^n$. Because $f^{-1}([0, 1])$ is closed and bounded in $\mathbb{R}^n$, it is sequentially compact.
\end{proof}

\subsection*{Problem 4 (Exercise 8, Page 304)}
Let $A$ and $B$ be sequentially compact subsets of $\mathbb{R}$. Define $K = \{(x, y) \text{ in } \mathbb{R}^2 \mid x \text{ in } A,\, y \text{ in } B\}$. Prove that $K$ is sequentially compact.

\begin{proof}
Take any arbitrary ${u_k} \in K$. We have that each $u_k = (x_k, y_k)$, with each $\{x_k\} \in A$ and each $\{y_k\} \in B$. Because $A$ is sequentially compact, there exists a subsequence $\{x_{k_j}\} \to x \in A$. Now consider $\{y_{k_j}\}$; because $B$ is sequentially compact, there exists a subsequence $\{y_{k_{j_l}}\} \to y \in B$. Because $\{x_{k_{j_l}}\} \to x$ as well (any subsequence of a convergent sequence converges to the same value), we use the component-wise convergence criterion to deduce that $\{u_{k_{j_l}}\} \to (x, y) = u \in K$, so that $K$ is sequentially compact.  

\end{proof}

\subsection*{Problem 5}
Theorem 0.1. \textit{Let $A \subseteq \mathbb{R}^n$ and consider $F \colon A \to \mathbb{R}^m$ a continuous mapping. If $A$ is compact then $F$ is uniformly continuous on $A$.} \\
Recall that by definition, $F$ is uniformly continuous on $A$ provided that for every $\varepsilon > 0$, there exists $\delta > 0$ such that for all $u, v \in A$, if $||u-v|| < \delta$ then $||F(u) - F(v)|| < \varepsilon$. \\
The purpose of this problem is to prove the preceding theorem in the following steps.
\begin{enumerate}
    \item Assume by contradiction that $F$ is not uniformly continuous on $A$. Show that there exist $\varepsilon_0 > 0$ and sequences $\{u_k\}$ and $\{v_k\}$ in $A$ such that
    \[ ||u_k - v_k|| < \frac{1}{k} \quad \text{and} \]
    \[ ||F(u_k) - F(v_k)|| > \varepsilon_0\]
    \item Use the compactness of $A$ to obtain a subsequence $\{u_{k_l}\}$ converging to some $u \in A$. Then show that $\{v_{k_l}\}$ also converges to $u$.
    \item Use the continuity of $F$ to derive a contradiction to (0.2).
\end{enumerate}

\begin{proof}
\quad \\
\begin{enumerate}
    \item Assume that $F$ is not uniformly continuous. Then there exists $\varepsilon > 0$ such that for all $\delta > 0$, there exist $x, y \in A$ such that $|x-y| < \delta$ but $|F(x)-F(y)| \geq \varepsilon$. In particular, set $\delta = 1/k$ for each natural number $k$, and let $\{u_k\}$ be the sequence of $x$ values that satisfy the first inequality and $\{v_k\}$ be the sequence of $y$ values that satisfy the first inequality. Letting $\varepsilon_0 = \varepsilon/2$ gives us the second strict inequality.
    \item Because $A$ is compact, by definition $\{u_k\}$ has a convergent subsequence $\{u_{k_l}\}$ converging to some $u \in A$, so that for all $\varepsilon > 0$ we have that there exists $L \in \mathbb{N}$ such that for all $l \geq L$, it follows that $||u_{k_l} - u|| < \frac{\varepsilon}{2}$. By the triangle inequality, we have that for $k_l \geq \frac{2}{\varepsilon}$, it follows that $||v_{k_1} - u|| = ||v_{k_1} - u_{k_l} + u_{k_l} - u|| \leq ||v_{k_1} - u_{k_l}|| + ||u_{k_l} - u|| < \frac{1}{k_l} + \frac{\varepsilon}{2} = \frac{\varepsilon}{2} + \frac{\varepsilon}{2} = \varepsilon$, so that $\{v_{k_1}\}$ also converges to $u$.
    \item Because $F$ is continuous, it follows that if $\{u_{k_l}\}$ converges to $u$, then $\{F(u_{k_l})\}$ converges to $F(u)$. Since $\{v_{k_l}\}$ also converges to $u$, it follows that $\{F(v_{k_l})\}$ also converges to $F(u)$. Note that for all $\varepsilon > 0$, there exists $N \in \mathbb{N}$ such that for all $n \geq N$, we have that $||F(u_{k_1}) - F(v_{k_l})|| = ||F(u_{k_l}) - F(u) + F(u) - F(v_{k_l})|| \leq ||F(u_{k_l}) - F(u) || + ||F(u) - F(v_{k_l})|| < \frac{\varepsilon}{2} + \frac{\varepsilon}{2} = \varepsilon$, which contradicts the fact that for every index $k$, $||F(u_k) - F(v_k)|| > \varepsilon_0$.
\end{enumerate} 
\end{proof}

\subsection*{Problem 6 (Problem 4, Page 309)}
Let $A$ and $B$ be convex subsets of $\mathbb{R}^n$. Prove that the intersection $A \cap B$ is also convex. Is it true that the intersection of two pathwise-connected subsets of $\mathbb{R}^n$ is also pathwise-connected?
\begin{proof}
Suppose that there exist $u, v \in A \cap B$; we want to show that there exists a line segment $\{tu + (1-t)v \mid 0 \leq t \leq 1\} \subseteq A \cap B$. Since $u, v \in A \cap B$, in particular $u, v \in A$ and $u, v \in B$. By assumption, $A$ is convex, so that there exists a line segment joining $u$ and $v$ $\{tu + (1-t)v \mid 0 \leq t \leq 1\} \subseteq A$. By assumption, $B$ is convex too, so that there exists a line segment joining $u$ and $v$ $\{tu + (1-t)v \mid 0 \leq t \leq 1\} \subseteq B$. Since $\{tu + (1-t)v \mid 0 \leq t \leq 1\} \subseteq A$ and $\{tu + (1-t)v \mid 0 \leq t \leq 1\} \subseteq B$, it follows that $\{tu + (1-t)v \mid 0 \leq t \leq 1\} \subseteq A\cap B$, so that for any arbitrary points $u, v \in A \cap B$ there exists a line segment joining them within $A \cap B$; thus $A \cap B$ is convex. \\
No, the intersection of two pathwise-connected subsets of $\mathbb{R}^n$ is not necessarily also pathwise-connected. Take $A = \{(x, y) \in \mathbb{R}^2 \mid x = 0\}$ and $B = \{(x, y) \in \mathbb{R}^2 \mid x^2+y^2 = 1\}$. Then each of $A$ and $B$ are pathwise-connected, because $A$ is a line and $B$ is the unit circle ($B = \{(\cos{(t)}, \sin{(t)}) \mid 0 \leq t \leq 2\pi\}$), but the intersection $A \cap B$ consists of only the two points $(-1, 0)$ and $(1, 0)$, which is not pathwise-connected in $\mathbb{R}^2$.

\end{proof}

\subsection*{Problem 7 (Problem 6, Page 309)}
Show that the set $S = \{(x, y) \text{ in } \mathbb{R}^2 \mid \text{either $x$ or $y$ is rational}\}$ is pathwise-connected.
\begin{proof}
It suffices to show that any point $(x, y) \in S$ can be brought to the origin with a path contained in the set, as any path from another arbitrary point $(x_1, y_1)$ to the origin can be reversed to get a path from the $(0, 0)$ to $(x_1, y_1)$. This means that we can get a path from $(x, y)$ to $(0, 0)$ to $(x_1, y_1)$. The only restriction is that $x$ and $y$ are not both irrational. To this end, we break this into cases: consider the case where initially $x$ is rational. Then there exists a straight line-segment path from $(x, y)$ to $(x, 0)$ because there is no restriction on $y$. Because $0$ is rational, there is now no restriction on $x$, so that there exists a straight line-segment path from $(x, 0)$ to $(0, 0)$. Now consider the case where initially $y$ is rational. Then there exists a straight line-segment path from $(x, y)$ to $(0, y)$. Because $0$ is rational, there is now no restriction on $y$, so that there exists a straight line-segment path from $(0, y)$ to $(0, 0)$.
\end{proof}

\end{document}
