\documentclass{article}
\usepackage[utf8]{inputenc}
\usepackage[english]{babel}
\usepackage[]{amsthm}
\usepackage[]{amsmath} 
\usepackage[]{amssymb} 

\title{HW1 - MATH411}
\author{Danesh Sivakumar}
\date{January 27, 2022}

\begin{document}
\maketitle 


\subsection*{Problem 1}

Find the maximum value of 

\[\frac{x^2+2y+3z}{\sqrt{x^2+y^2+z^2}}\]

as $(x, y, z)$ varies among nonzero points in $\mathbb{R}^3$

\begin{proof}

Let $\vec{u} = (x, y, z)$ and $\vec{v} = (1, 2, 3)$. Then, by Cauchy-Schwarz:

\[\frac{x^2+2y+3z}{\sqrt{x^2+y^2+z^2}} = \frac{\langle \vec{u},\vec{v}\rangle}{||\vec{v}||} \leq \frac{||\vec{u}|| ||\vec{v}||}{||\vec{u}||} = ||\vec{v}|| = \sqrt{14}\]

So $\sqrt{14}$ is an upper bound for the expression, and it is attained when $\vec{u} = \lambda \vec{v}$; taking $\lambda = 1$ yields $\vec{u} = \vec{v}$, so $\vec{u} = (1, 2, 3)$. Then:

\[ \frac{x^2+2y+3z}{\sqrt{x^2+y^2+z^2}} = \frac{1(1)+2(2)+3(3)}{\sqrt{1^2+2^2+3^2}} = \frac{14}{\sqrt{14}} = \sqrt{14}\]

Thus $\sqrt{14}$ is the maximum value.


\end{proof}


\subsection*{Problem 2}
Let $\textbf{u}$ and $\textbf{v}$ be vectors in $\mathbb{R}^n$. Prove that
\[ \langle \textbf{u}, \textbf{v} \rangle = \frac{||\textbf{u} + \textbf{v}||^2 - ||\textbf{u} - \textbf{v}||^2}{4}\]
\begin{proof}

By the expansion formula covered in class

\[ ||\textbf{u} + \textbf{v}||^2 = ||\textbf{u}||^2 + ||\textbf{v}||^2 + 2\langle \textbf{u}, \textbf{v} \rangle \]

We deduce that

\[\frac{||\textbf{u} + \textbf{v}||^2 - ||\textbf{u} - \textbf{v}||^2}{4} = \frac{(||\textbf{u}||^2 + ||\textbf{v}||^2 + 2\langle \textbf{u}, \textbf{v} \rangle) - |\textbf{u} + \textbf{-v}||^2 }{4}\]

\[ = \frac{(||\textbf{u}||^2 + ||\textbf{v}||^2 + 2\langle \textbf{u}, \textbf{v} \rangle) - (||\textbf{u}||^2 + ||\textbf{v}||^2 + 2\langle \textbf{u}, \textbf{-v} \rangle)}{4}\]

\[ = \frac{(||\textbf{u}||^2 + ||\textbf{v}||^2 + 2\langle \textbf{u}, \textbf{v} \rangle) - (||\textbf{u}||^2 + ||\textbf{v}||^2 - 2\langle \textbf{u}, \textbf{v} \rangle)}{4}\]

\[ = \frac{4\langle \textbf{u}, \textbf{v} \rangle}{4} = \langle \textbf{u}, \textbf{v} \rangle \]

\end{proof}

\subsection*{Problem 3}
For points $\textbf{u}$ and $\textbf{v}$ in $\mathbb{R}^n$, define the function $p \colon \mathbb{R} \to \mathbb{R}$ by $p(t) = ||\textbf{u} + t\textbf{v}||^2$ for $t$ in $\mathbb{R}$. Show that $p(t)$ is a quadratic polynomial that attains only nonnegative values. Use this to show that the discriminant is nonpositive and thus provide another proof of the Cauchy-Schwarz Inequality.

\begin{proof}

By definition of norm, we have that

\[ ||\textbf{u} + t\textbf{v}||^2 \geq 0\].

Using the formula from the previous problem mentioned in class, we have that

\[ ||\textbf{u} + t\textbf{v}||^2 = ||\textbf{u}||^2 + ||t\textbf{v}||^2 + 2\langle \textbf{u}, t\textbf{v} \rangle = t^2||\textbf{v}||^2 + 2t\langle \textbf{u}, t\textbf{v} \rangle + ||\textbf{u}||^2\]

which is a polynomial in $t$ that only attains nonnegative values. Because the polynomial only attains nonnegative values, it either has a double real root or no real roots, meaning that the discriminant $b^2 - 4ac \leq 0$. Letting $a=||\textbf{v}||^2$, $b=2\langle \textbf{u}, \textbf{v}\rangle$, and $c=||\textbf{u}||^2$, this is equivalent to:

\[(2\langle \textbf{u}, \textbf{v}\rangle)^2 - 4||\textbf{u}||^2||\textbf{v}||^2 \leq 0\]

\[ \iff (2\langle \textbf{u}, \textbf{v}\rangle)^2 \leq 4||\textbf{u}||^2||\textbf{v}||^2\]

\[ \iff 2\langle \textbf{u}, \textbf{v}\rangle \leq 2||\textbf{u}||||\textbf{v}||\]

\[ \iff \langle \textbf{u}, \textbf{v}\rangle \leq ||\textbf{u}||||\textbf{v}||\]

which is the Cauchy-Schwarz inequality.

\end{proof}

\subsection*{Problem 4}
Suppose that the points $\mathbf{u_1}, \cdots, \mathbf{u_k}$ in $\mathbb{R}^n$ are an orthonormal set. For $\mathbf{u} = \alpha_1\mathbf{u_1} + \cdots + \alpha_k\mathbf{u_k}$, show that

\[ ||\mathbf{u}|| = \sqrt{\sum_{i=1}^{k} \alpha_i^2} \]

\begin{proof}

Note that $||\mathbf{u}|| = ||\alpha_1\mathbf{u_1} + \cdots + \alpha_k\mathbf{u_k}||$, so that $||\mathbf{u}||^2 = ||\alpha_1\mathbf{u_1} + \cdots + \alpha_k\mathbf{u_k}||^2$. We first prove that $||\alpha_1\mathbf{u_1} + \cdots + \alpha_k\mathbf{u_k}||^2 = ||\alpha_1\mathbf{u_1}||^2 + \cdots + ||\alpha_k\mathbf{u_k}||^2$. by induction\\ 

To this end, we show that the base case holds. Because each of the $\mathbf{u_k}$ are orthonormal, each of the $\alpha_k\mathbf{u_k}$ are orthogonal, and from a theorem in class we deduce that $||\alpha_1\mathbf{u_1} + \alpha_2\mathbf{u_2}||^2 = ||\alpha_1\mathbf{u_1}||^2 + ||\alpha_2\mathbf{u_2}||^2$. Supposing that $||\alpha_1\mathbf{u_1} + \cdots + \alpha_k\mathbf{u_k}||^2 = ||\alpha_1\mathbf{u_1}||^2 + \cdots + ||\alpha_k\mathbf{u_k}||^2$, we aim to show $||\alpha_1\mathbf{u_1} + \cdots + \alpha_k\mathbf{u_k} + \alpha_{k+1}\mathbf{u_{k+1}}||^2  = ||\alpha_1\mathbf{u_1}||^2 + \cdots + ||\alpha_k\mathbf{u_k}||^2 + ||\alpha_{k+1}\mathbf{u_{k+1}}||^2$. Note that $\alpha_1\mathbf{u_1} + \cdots + \alpha_k\mathbf{u_k}$ is orthogonal to $\alpha_{k+1}\mathbf{u_{k+1}}$, so that

\[ ||(\alpha_1\mathbf{u_1} + \cdots + \alpha_k\mathbf{u_k}) + \alpha_{k+1}\mathbf{u_{k+1}}||^2 \]
\[ = ||\alpha_1\mathbf{u_1} + \cdots + \alpha_k\mathbf{u_k}||^2 + ||\alpha_{k+1}\mathbf{u_{k+1}}||^2 \]
\[ = ||\alpha_1\mathbf{u_1}||^2 + \cdots + ||\alpha_k\mathbf{u_k}||^2 + ||\alpha_{k+1}\mathbf{u_{k+1}}||^2\].

By properties of norms:

\[ ||\alpha_1\mathbf{u_1} + \cdots + \alpha_k\mathbf{u_k}||^2 = ||\alpha_1\mathbf{u_1}||^2 + \cdots + ||\alpha_k\mathbf{u_k}||^2\]
\[= \alpha_1^2||\mathbf{u_1}||^2 + \cdots + \alpha_k^2||\mathbf{u_k}||^2 = \sum_{i=1}^{k} \alpha_i^2\mathbf{u_i}^2 = \sum_{i=1}^{k} \alpha_i^2(1)^2 = \sum_{i=1}^{k} \alpha_i^2\].

Taking the square root of both sides gives the result.

\end{proof}

\subsection*{Problem 5}
Let $\mathbf{\{u_k\}}$ be a sequence in $\mathbb{R}^n$ that converges to the point $\mathbf{u}$. Prove that 

\[ \lim_{k\to\infty} \langle \mathbf{u_k}, \mathbf{v} \rangle = \langle \mathbf{u}, \mathbf{v} \rangle\]

\begin{proof}

Because $\mathbf{\{u_k\}}$ converges to $\mathbf{u}$, it follows that $\mathbf{\{u_k\}}$ converges component-wise to $\mathbf{u}$, so that:

\[ \lim_{k\to\infty} \langle \mathbf{u_k}, \mathbf{v} \rangle = \lim_{k\to\infty} [p_1(\mathbf{u_k})\mathbf{v_1} + \cdots + p_n(\mathbf{u_k})\mathbf{v_n}] = p_1(\mathbf{u})\mathbf{v_1} + \cdots + p_n(\mathbf{u})\mathbf{v_n}\]

\[ = \mathbf{u_1}\mathbf{v_1} + \cdots \mathbf{u_n}\mathbf{v_n} = \langle \mathbf{u}, \mathbf{v} \rangle \]

\end{proof}

\subsection*{Problem 6}
Let $\mathbf{\{u_k\}}$ be a sequence in $\mathbb{R}^n$ and let $\mathbf{u}$ be a point in $\mathbb{R}^n$. Suppose that for every $\mathbf{v}$ in $\mathbb{R}^n$,

\[ \lim_{k\to\infty} \langle \mathbf{u_k}, \mathbf{v} \rangle = \langle \mathbf{u}, \mathbf{v} \rangle.\]

Prove that $\mathbf{\{u_k\}}$ converges to $\mathbf{u}$.

\begin{proof}

Let $\{e_1, \cdots e_i\} \subset \mathbb{R}^n$ be the set of vectors where each $e_i$ is the vector whose $i$-th component is 1 and the others are all 0. Because 

\[ \lim_{k\to\infty} \langle \mathbf{u_k}, \mathbf{v} \rangle = \langle \mathbf{u}, \mathbf{v} \rangle.\]

holds for every $\mathbf{v}$, we have that

\[ \lim_{k\to\infty} \langle \mathbf{u_k}, e_i \rangle = \langle \mathbf{u}, e_i \rangle\]

so that for each $i \in \{1, \cdots, n\}$

\[ \lim_{k\to\infty} p_i(\mathbf{u_k}) = p_i(\mathbf{u}) \]

which implies that each component of $\mathbf{\{u_k\}}$ converges to the respective component in $\mathbf{u}$, so that by the component-wise convergence theorem, $\mathbf{\{u_k\}}$ converges to $\mathbf{u}$.

\end{proof}

\subsection*{Problem 7}
Suppose that $\mathbf{\{u_k\}}$ is a sequence of points in $\mathbb{R}^n$ that converges to the point $\mathbf{u}$ and that $||\mathbf{u}|| = r > 0$. Prove that there is an index $K$ such that
\[ ||\mathbf{u_k}|| > r/2 \qquad \text{ if } k \geq K\]

\begin{proof}

Because $\mathbf{u_k}$ converges to $\mathbf{u}$, by a theorem in class it follows that $\lim_{k\to\infty} || u_k || = || u ||$. By convergence of real sequences, for all $\epsilon > 0$, there exists $K \in \mathbb{N}$ such that

\[ k \geq K \implies | ||u_k||-||u|| | <  \epsilon \]

Letting $\epsilon = r/2$, we observe that because $u = r$

\[ -r/2 < ||u_k|| - r < r/2 \implies r/2 < ||u_k|| < 3r/2\]

The left side of the inequality shows that for all $k \geq K$, we have that $r/2 < ||u_k||$.

\end{proof}

\subsection*{Problem 8}
Let $\{u_k\}_{k \geq 1}$ be a sequence in $\mathbb{R}^n$ and $\{a_k\}_{k \geq 1}$ be a sequence in $\mathbb{R}$. Prove that if

\[ \lim_{k\to\infty} u_k = u \in \mathbb{R}^n \quad \text{ and } \quad \lim_{k\to\infty} a_k = a \in \mathbb{R} \]
then
\[ \lim_{k\to\infty} (a_ku_k) = au \]

\begin{proof}

Because $\lim_{k\to\infty} u_k = u$, $u_k$ converges component-wise, so that $\lim_{k\to\infty} p_i(u_k) = p_i(u)$. This means for each $i \in \{1, \cdots, n\}$

\[ \lim_{k\to\infty} (a_ku_k) = \lim_{k\to\infty} [a_kp_1(u_k) + \cdots a_kp_n(u_k)]\]
\[ = ap_1(u) + \cdots + ap_n(u) = a \sum_{i=1}^{n} p_i(u) = au\]

\end{proof}


\end{document}
